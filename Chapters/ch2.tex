\section{Basic Topology}
\subsection{Finite and Countable Sets}
This chapter is split into two portions; the first looks at counting, what it means for us to say that two sets have the same number of elements, and concludes with a classic theorem of Cantor concerning uncountable sets. The second part looks at the topology of metric spaces, before moving onto the topology of the real numbers. 

Let us then begin with our discussion of counting. If we consider counting how many bananas there are on a table (say there are 10 bananas), then what we are formally doing is establishing a correspondence between each ball on the table with an element in the set $\set{1, \ldots, 10}$. When we refer to the number of elements in a set, it will be good to keep in mind that we are establishing functions between sets. Although we have been discussing functions with some frequency in the course already, we give a definition below for completeness. 

\begin{definition}{Functions}{2.1}
    Let $A, B$ be sets. Then, a map that associates each element $x \in A$ with a unique element denoted as $f(x) \in B$ is a \textbf{function} $f: A \rightarrow B$. We then define $A$ as the \textbf{domain} of $f$ and the set $\set{f(x): x \in A}$ as the \textbf{range}. For $E \subseteq A$, we call $f(E) = \set{f(x): x \in E}$ the \textbf{image} of $E$ under $f$. For $F \subseteq B$, we call $f^{-1}(B) = \set{x \in A: f(x) \in B}$ the \textbf{preimage} of $F$. 
\end{definition}

\begin{definition}{Injective/Surjective Functions}{2.2}
    Let $f: A \mapsto B$ be a function. If for $x_1, x_2 \in A$ we have that $f(x_1) = f(x_2) \implies x_1 = x_2$ (or equivalently, $x_1 \neq x_2 \implies f(x_1) \neq f(x_2)$), then we say that $f$ is \textbf{injective}, or \textbf{one-to-one}. If for all $y \in B$ there exists $x \in A$ such that $y = f(x)$, then we say that $f$ is \textbf{surjective}, or \textbf{onto}. If a function is both injective and surjective, it is \textbf{bijective}.
\end{definition}
\noindent Intuitively, we can think of injectivity as implying each element in $B$ being reached at most once, and surjectivity implying that each element in $B$ is reached at least once. 

\begin{definition}{Cardinality \& Equivalence}{2.3}
    Let $A, B$ be sets. We say that $A, B$ have the same \textbf{cardinality} if there exists $f: A \mapsto B$ such that $f$ is bijective. We can denote this as $A \sim B$ where $\sim$ indicates an \textbf{equivalence relation}. An equivalence relation has three properties:
    \begin{enumerate}
        \item Reflexivity: $A \sim A$.
        \item Symmetry: If $A \sim B$ then $B \sim A$.
        \item Transitivity: If $A \sim B$ and $B \sim C$ then $A \sim C$.
    \end{enumerate}
    As a point of notation, $\abs{S}$ denotes the cardinality of the set $S$. 
\end{definition}

\noindent We get (a) from each set having a bijection to itself (i.e. the identity function), (b) from the fact that if there exists a bijection $f: A \mapsto B$, then there must exist an inverse $f^{-1}: B \mapsto A$, and (c) from if there exist bijections $f: A \mapsto B$ and $g: B \mapsto C$ then the composition $g \circ f: A \mapsto C$ will also be a bijection. 

\begin{definition}{Countability}{2.4}
    First, we denote $J_n = \set{1, 2, 3, \ldots, n}$ and $J = \NN = \set{1, 2, 3, \ldots}$. Let $A$ be a set. We say that $A$ is \textbf{finite} if it has a finite number of elements, that is, there exists $n \in \NN$ such that $A \sim J_n$. A set $A$ is \textbf{infinite} if it is not finite, and we cannot put $A$ in bijection with $J_n$ for any $n \in \NN$. We say that $A$ is \textbf{countable}if $A \sim \NN$, and \textbf{uncountable} otherwise. 
\end{definition}

\noindent Note that the above 

\begin{example}{}{2.5}
    $\ZZ$ is countable. To see this, consider the function:
    \begin{align*}
        f = \begin{cases}
            \frac{n}{2} & \text{$n$ is even}
            \\ -\frac{n-1}{2} & \text{$n$ is odd}
            \end{cases}
    \end{align*}
    $f$ is a bijection (check!) and hence $\NN \sim \ZZ$. 
\end{example}
\noindent The above example serves as a bit of a warning sign. Even though $\NN \subsetneq \ZZ$ and $\ZZ$ has ``more elements'', we still find that the two sets have the same cardinality. 
\setcounter{rudin}{7}

\begin{theorem}{}{2.8}
    A subset of a countable set is either finite or countable.
\end{theorem}
\begin{nproof}
    (Sketch) The countability of $A$ implies that $A = \set{a_1, a_2, a_3, a_4, a_5 \ldots}$ (in other words, we can enumerate the elements using $\NN$). Let $S \subseteq A$. Then, $S = \set{a_1, \cancel{a_2}, a_3, a_4, \cancel{a_5}, \ldots}$, that is, $A$ with some (or none) of the elements removed. Now, we can rename all the elements with $a_1, a_2, \ldots$; what we have left is again an enumeration, so it is yet again (at most) countable.
\end{nproof}
\noindent One potentially useful fact is that if we have a set $S$ and a function $f: \NN \mapsto S$ such that $f$ is surjective, then $S$ is at most countable. 

\begin{proof}
    Let $T = \set{n \in \NN: f(n) \neq f(m), \forall m = 1, 2, \ldots, m}$. We restrict $f: T \mapsto S$, then $f$ is injective by constructive. It is still surjective, hence $T \sim S$. Since $T \subset \NN$, by Theorem \ref{thm:2.8}, $S$ is finite or countable.
\end{proof}

\setcounter{rudin}{11}

\begin{theorem}{}{2.12}
    Let $E_1, E_2, \ldots$ be countable sets (i.e. we have a countable number of countable sets). Define $S = \bigcup_{n=1}^{\infty} E_n$. Then, $S$ is countable. 
\end{theorem}
\begin{nproof}
    Write $E_n = \set{x_{n}^1, x_{n}^2, x_{n}^3, \ldots}$ (which we can do as each of the $E_n$s are countable). Then, we form an array:
    \begin{center}
    \begin{tikzpicture}
        \matrix[matrix of math nodes,inner sep=1pt,row sep=1em,column sep=1em] (M)
        {
            E_1 & = & x_{1}^{1} & x_{1}^{2} & x_{1}^{3}  & \cdots \\
            E_2 & = & x_{2}^{1} & x_{2}^{2} & x_{2}^{3}  & \cdots \\
            E_3 & = & x_{3}^{1} & x_{3}^{2} & x_{3}^{3}  & \cdots \\
            \cdots \\
        }
        ;
        \draw[->] (M-1-3.south west) -- (M-1-3.north east);
        \draw[->] (M-2-3.south west) -- (M-1-4.north east);
        \draw[->] (M-3-3.south west) -- (M-1-5.north east);
        \draw[->] (M-1-3.south west) -- (M-1-3.north east);
        \draw[->] (M-3-4.south west) -- (M-2-5.north east);
        \draw[->] (M-3-5.south west) -- (M-3-5.north east);
    \end{tikzpicture}
    \end{center}
    Then, we can re-number the elements along the diagonal lines (i.e. $x_1^1, x_2^1, x_1^2, x_3^1, x_2^2, x_1^3, \ldots$). This new enumeration corresponds to a countable set. From there, we let $T \subseteq \NN$ be the remaining labels in the enumeration after removing the repeated elements from the sequence. Then, $T \sim S$, and hence $S$ is at most countable. $S$ cannot be finite as $E_1 \subseteq S$ and $E_1$ is not finite. Hence $S$ is countable. \qed
\end{nproof}

\begin{corollary}{}{2.13}
    \begin{itemize}
        \item If $A$ is countable, the set of n-tuples of $(a_1, \ldots a_n)$ is also countable for any $n \in \NN$.  
        \item $\QQ$ is countable.
    \end{itemize}
\end{corollary}
\noindent We defined $\QQ$ as pairs of integers, but by the first part of the corollary (which follows immediately by application of Theorem \ref{thm:2.12}) $\ZZ^2$ (the set of pairs of integers) has equal cardinality to $\ZZ$, and since $\QQ$ is a subset of the set of pairs of integers, $\QQ$ is countable. 

From the discussion of today, we have established that $\abs{\NN} = \abs{\ZZ} = \abs{\QQ}$. Does $\RR$ also have equal cardinality to these sets? The answer turns out to be no, as we will see shortly. 


\subsection{Uncountable Sets}
\subsection{Topology of Metric Spaces}
\subsection{Closure and Relative Topology}
\subsection{Compactness}
\subsection{Compactness in \texorpdfstring{$\RR^k$}{TEXT} and the Cantor Set}
